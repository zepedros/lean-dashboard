\documentclass{article}
\usepackage[utf8]{inputenc}
\usepackage{hyperref}
\usepackage{enumitem}
\newlist{steps}{enumerate}{1}
\setlist[steps, 1]{label = Step \arabic*:}


\title{\huge\textbf{Lean Dashboard - Organization Description}}



\begin{document}
\maketitle

\textbf{Authors}:
\begin{center}
\begin{tabular}{ c }
 José Pedro Jesus, n.º 44805 \\
 Hugo Manuel Jacinto Pinheiro, n.º 44886 \\
 Tomás Simão Mendes dos Santos, n.º 45363
\end{tabular}
\end{center}

\textbf{Supervisors}:´
\begin{center}
\begin{tabular}{ c }
    João Pereira, e-mail: joao.pereira@inetum.world, Inetum\\
    Filipe Freitas, e-mail: ffreitas@cc.isel.ipl.pt\\
\end{tabular}
\end{center}


\textbf{Github Repository}: \url{https://github.com/zepedros/lean-dashboard}

\section{Application Back-end}
In order to run the Back-end of the application it is first necessary to have installed NodeJS, PostgreSQL, Postman and Elasticsearch. After that, it should be possible to run the application by following these instructions:

\begin{center}
\begin{steps}
  \item Run Elasticsearch on your machine
  \item Setup .env file for the database. This file should be located in \path{lean-dashboard\code\js}. In this file you setup details of the database being used such as the host, port, user and password (see our example below)
  \item Run PostgreSQL service on your machine
  \item Open Postman and import the collection for the API located on \path{lean-dashboard\code\Lean Dashboard.postman_collection.json}
  \item Open directory \path{lean-dashboard\code\js}
  \item Open the terminal and run the command \textbf{npm install package.json}
  \item Run the command \textbf{npm \path{lean\lean-server.js}} to start the application
 
\end{steps}
\end{center}

\newpage
Example of .env file:\\
\newline
HOST= localhost \\
DB\_PORT= 5432 \\
DB\_USER= postgres \\
DB\_PASSWORD= 1234 \\
DB\_CONNECTION\_LIMIT= 5 \\
DATABASE= authization \\
DBMS= postgres \\

\section{Application Front-end}
Some of the pages for the application have already begun being developed, without the use of the API developed in the back-end. These can be tested with the help of the following instructions:

\begin{center}
\begin{steps}
  \item Open directory \path{lean-dashboard\code\leandashboard-ui}
  \item Open the terminal and run the command \textbf{npm install package.json}
  \item Run the command \textbf{npm start}
  \item Wait for your browser to open a window with the application, or access \url{http://localhost:3000/} to open the application manually
 
\end{steps}
\end{center}
\end{document}
